\documentclass{article}
\usepackage{polski}
\begin{document}

\section*{1. Tytuł projektu: Bomber}

\section*{2. Skład grupy}
\begin{itemize}
    \item Szymon Gaczoł (grupa 1);
    \item Jakub Wolny (grupa 1);
    \item Olgierd Zygmunt (grupa 1).
\end{itemize}
\section*{3. Gra Bomber}

Gra Bomber to rozrywka przeznaczona dla 1-4 graczy. Każdy z uczestników stara się wyeliminować pozostałych poprzez strategiczne umieszczanie bomb na labiryntowej planszy. Jeśli gracz będzie znajdował się wystarczająco blisko bomby i nie będzie osłonięty przez elementy planszy, zostanie zraniony, a jego pasek zdrowia obniży się. Gdy zdrowie spadnie poniżej zera, gracz zostaje usunięty z gry. Zwycięzcą zostaje ten, kto utrzyma się na planszy jako ostatni z punktami zdrowia.

\subsection*{Możliwe Rozszerzenia:}
\begin{itemize}
    \item Losowo generowane (sensowne) mapy;
    \item Różne rodzaje bomb pojawiające się co jakiś czas do podniesienia;
    \item Tryb 2 vs 2;
    \item Różne rodzaje trudności w grze przeciwko komputerowi;
    \item Zmieniający się teren;
    \item Pojawiające się apteczki.
\end{itemize}

\section*{4. Biblioteki/Frameworki}
\begin{itemize}
    \item libGDX
\end{itemize}
\section*{5. Repozytorium}
https://github.com/olcio69/bomber
\end{document}

